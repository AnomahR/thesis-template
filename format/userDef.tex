%%%%%%%%%%%%%%%%%%%%%%%%%%%%%%%%%%%%%%%%%%%%%%%%%%%%
% TO CHANGE
% path and extension of graphics, to be continued
\graphicspath{{./images/}}
\DeclareGraphicsExtensions{.eps}

% author, title, date of thesis 
\newcommand*{\Title}{Your Title}
\newcommand*{\Autor}{Your Name}
%\newcommand*{\Datum}{Berlin, \today}
\newcommand*{\Datum}{Your Date}
\title{\Title}
\author{\Autor} 
\date{\Datum}

% Beispiel f�r Definition neuer Kommandos f�r h�ufig gebrauchte Konstrukte
\newcommand{\tfk}[1]{\textsl{\texttt{#1}}}
% correct spacing in align+array environment (... \arrcor & = ...)
\newcommand{\arrcor}{\! \! \! \!}

\DeclareMathOperator{\largevarepsilon}{\mathlarger{\mathlarger{\varepsilon}}}
\DeclareMathOperator{\RE}{Re}
\DeclareMathOperator{\IM}{Im}
\DeclareMathOperator{\rank}{rank}
\DeclareMathOperator{\spanop}{span}
\DeclareMathOperator{\diag}{diag}
% f�r Standardumgebungen
%\renewcommand{\labelitemi}{*}												% Aufz�hlungszeichen definieren

% Mathematical environments
\theoremstyle{plain}
\newtheorem{theorem}{Theorem}[chapter]
\newtheorem{lemma}[theorem]{Lemma}
\newtheorem{proposition}[theorem]{Proposition}
\newtheorem{corollary}[theorem]{Corollary}
\newtheorem{observation}[theorem]{Observation}

\theoremstyle{definition}
\newtheorem{problem}[theorem]{Problem}
\newtheorem{definition}[theorem]{Definition}

\theoremstyle{remark}
\newtheorem{remark}[theorem]{Remark}

% scalebox for math
\newcommand{\scaleboxfix}[1]{\scalebox{0.7}{#1}}